\documentclass{scrartcl}
\usepackage[english,ngerman]{babel}
\usepackage[utf8]{inputenc}                
\usepackage[T1]{fontenc}  
\usepackage[ngerman]{babel}
\usepackage{lmodern}

\begin{document}
% ----------------------------------------------------------------------------
% Titel 
%\titlehead{Kopf über dem Titel mit Leerstuhl u.\,ä.}% optional
%\subject{Art des Dokuments}% optional
\title{Titel des Dokuments}% obligatorisch
\subtitle{Untertitel}% optional
\author{Dr.-Ing. Robert Hilbrich}% obligatorisch
\date{}% sinnvoll
%\publishers{Platz für Betreuer o.\,ä.}% optional
\maketitle% verwendet die zuvor gemachte Angaben zur Gestaltung eines Titels
% ----------------------------------------------------------------------------
% Inhaltsverzeichnis:
%\tableofcontents
% ----------------------------------------------------------------------------
% Gliederung und Text:

\section{Motivation}
\label{sec:motivation}
Dieser Abschnitt sollte sich mit der Aufgabenstellung befassen. Er kann auch
Grundlagen behandeln. Es kann jedoch sinnvoll sein, für die Grundlagen einen
eigenen Abschnitt zu verwenden.

\section{Durchführung}
\label{sec:durchfuehrung}
Hier erzählt man nun, was man alles gemacht hat.

\section{Schluss}
\label{sec:schluss}
Hierher gehört das Fazit und ggf. der Ausblick auf weitere Dinge, die getan
werden könnten.

\end{document}
%%% Local Variables:
%%% mode: latex
%%% TeX-master: t
%%% End:
