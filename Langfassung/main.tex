\documentclass[parskip=half]{scrartcl}
\usepackage[english,ngerman]{babel}
\usepackage[latin1]{inputenc}                
\usepackage[T1]{fontenc}  
\usepackage{lmodern}

\begin{document}
% ----------------------------------------------------------------------------
% Titel 
%\titlehead{Deutsches Zentrum f�r Luft- und Raumfahrt e.V.}
%\subject{Art des Dokuments}
\title{Automatisierung und Kreativit�t}
%\subtitle{Untertitel}% optional
\author{Robert Hilbrich}
\date{}
%\publishers{Platz f�r Betreuer o.\,�.}
\maketitle
% ----------------------------------------------------------------------------
% Inhaltsverzeichnis:
%\tableofcontents
% ----------------------------------------------------------------------------
% Gliederung und Text:

\section{Einf�hrung}
\label{sec:einfuhrung}

Das Ende der �ra der Lochkarten zur manuellen Programmierung war zugleich auch der Beginn eines deutlich erweiterten Aufgabenspektrums der Computersysteme.
Neben ihrer Rolle als Laufzeitplattform und Ressourcenlieferant zur Ausf�hrung von manuell entwickelten Codes, konnten sie im Zuge ihrer zunehmenden Leistungsf�higkeit auch die \emph{Entwicklung} von Code unterst�tzen.
Mittlerweile ist die Softwareentwicklung ohne die Unterst�tzung von Computern nicht mehr sinnvoll durchzuf�hren. 
Computer transformieren manuell entwickelten Code aus Hochsprachen in Maschinensprache, sie analysieren die Codes auf Speicherlecks oder ung�ltige Ressourcenzugriffe und validieren deren Korrektheit mit Hilfe einer Vielzahl von Testf�llen "= alles weitgehend automatisch.

Nicht nur bei der Entwicklung von Software konnten sich Computersysteme als hilfreiche Assistenten emanzipieren.
Mittlerweile ist auch die Entwicklung von Hardware zwingend auf den Einsatz von Computern angewiesen.
So werden die komplexen Steuerungs"= und Assistenzsysteme in Fahrzeugen und Flugzeugen beispielsweise nur noch mit Hilfe von Computersystemen entworfen.
Dies ist notwendig, um die Komplexit�t der zu entwickelnden Systeme zu beherrschen und zugleich auch �konomischen Randbedingungen zu gen�gen.

Obwohl sich die Leistungsf�higkeit der Rechentechnik und damit auch ihr Nutzen bei der der Unterst�tzung von Entwicklungst�tigkeiten seit den Lochkarten deutlich erh�ht hat, ist die grundlegende Arbeitsteilung zwischen dem Menschen und der Maschine erhalten geblieben.
W�hrend der menschliche Entwickler die \emph{kreativ"=konstruktiven} T�tigkeiten bei der Synthese von Komponenten und Artefakten �bernimmt, sind die \emph{analytischen} und \emph{automatisierbaren} T�tigkeiten dem Computer vorbehalten.
So �bernimmt der Mensch die sch�pferischen T�tigkeiten, durch die er etwas Neues und Originelles erschafft. 
Die stupiden und repetitiven T�tigkeiten der Entwicklung werden dagegen durch den Computer �bernommen und von ihm entsprechend einer vordefinierten Art und Weise in identischer Manier ausgef�hrt.

Aufgrund dieser Arbeitsteilung ist die Komplexit�t der zu entwickelnden Systeme durch die menschliche Verarbeitungskapazit�t begrenzt, denn jedes Entwicklungsvorhaben ist zun�chst auf die Durchf�hrung von kreativ"=konstruktiven T�tigkeiten angewiesen.





\end{document}
%%% Local Variables:
%%% mode: latex
%%% TeX-master: t
%%% End:
